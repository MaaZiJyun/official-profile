\documentclass{article}
\usepackage{graphicx}
% latex.js may not parse \graphicspath reliably; comment it out and reference images by full or relative path
% \graphicspath{{/posts/Minimum_Constellation_Size_for_Polar_LEO_Global_Coverage/}}
\title{How many satellites and how much time are needed for full global coverage?}
\author{Zi Jyun Maa}
\date{September 2025}

\begin{document}
\maketitle

\section{Introduction}

Imagine we want to cover the entire Earth. How many orbital planes do we need? And how many satellites should we place in each plane?
Below is an exploration of how these quantities relate to Earth’s rotation, orbital motion, and coverage requirements.
Because we are considering about full global coverage, 90° inclination (polar) orbits is the only option.

\section{Why does Earth’s rotation matter?}

Even though a polar orbit (inclination $90^\circ$) sweeps across all latitudes, along the equator each orbital plane only covers a narrow strip with width $L_{\text{swath}}$.
Within a maximum allowed response time $t_{\text{max}}$, Earth rotates, causing an equatorial displacement:

\begin{equation}
	v_{\oplus} = \frac{2\pi R_{\oplus}}{T_{\oplus}}
\end{equation}
where $R_{\oplus}$ is Earth’s radius and $T_{\oplus}$ is Earth’s rotation period (about 24 hours).
So in time $t_{\text{max}}$, the ground track shifts by:

\begin{equation}
	S_{\text{equ}} = v_{\oplus} \cdot T_{\text{max}}
\end{equation}
where $S_{\text{equ}}$ is the maximum distance Earth’s rotation can cause the ground track to shift on equator within the allowed response time.
This means each orbital plane can only cover a strip of width $L_{\text{swath}}$ plus this shift:

\begin{figure}[ht]
	\centering
	% Use an absolute path from site root so browser-based renderers (latex.js) can fetch it:
	\includegraphics[width=0.8\textwidth]{/posts/Minimum_Constellation_Size_for_Polar_LEO_Global_Coverage/001.png}
	\caption{Swath coverage from one orbital plane, considering Earth’s rotation over time max.}
	\label{fig:constellation-example}
\end{figure}

So the first question becomes:

\textbf{How many such strips are needed so that Earth’s rotation never leaves a gap?}

This gives the minimum number of orbital planes:

\begin{equation}
n_{\text{orb}} = \frac{T_{\oplus}}{2 T_{\text{max}}}
\end{equation}
where $n_{orb}$ is the number of orbital planes needed.

\begin{figure}[ht]
	\centering
	% Use an absolute path from site root so browser-based renderers (latex.js) can fetch it:
	\includegraphics[width=0.8\textwidth]{/posts/Minimum_Constellation_Size_for_Polar_LEO_Global_Coverage/002.png}
	\caption{Swath coverage from many orbital planes, considering Earth’s rotation over time max.}
	\label{fig:constellation-example}
\end{figure}

\section{How fast does each orbit “move” to the next strip?}

To understand spacing along each orbit, we ask:
How long until Earth rotates enough that the ground track shifts by one swath width?

\begin{equation}
t_{\text{swath}} = \frac{L_{\text{swath}} T_{\oplus}}{2\pi R_{\oplus}}
\end{equation}
where $t_{\text{swath}}$ is the time it takes for Earth’s rotation to shift the ground track by one swath width.

And how far does a satellite travel in orbit during that time?

\begin{equation}
S_{\text{orb}} = \frac{R_{\text{orb}} L_{\text{swath}} T_{\oplus}}{T_{\text{orb}} R_{\oplus}}
\end{equation}
where $R_{\text{orb}}$ is the orbital radius and $T_{\text{orb}}$ is the orbital period.
This tells us the distance along the orbit that corresponds to one swath width on the ground.
To ensure continuous coverage, we need to place satellites along the orbit such that the spacing does not
exceed this distance.

This tells us the maximum spacing between satellites in the same plane.
So we get:

\begin{equation}
n_{\text{sats/orb}} = \frac{2\pi R_{\oplus} T_{\text{orb}}}{L_{\text{swath}} T_{\oplus}}
\end{equation}
where $n_{\text{sats/orb}}$ is the number of satellites needed per orbital plane.

\section{How many satellites in total?}

Just multiply orbital planes by satellites per plane:

\begin{equation}
n_{\text{sat}} = \frac{\pi R_{\oplus} T_{\text{orb}}}{L_{\text{swath}} T_{\text{max}}}
\end{equation}
where $n_{\text{sat}}$ is the total number of satellites needed for full global coverage within the specified response time.

\begin{figure}[ht]
	\centering
	% Use an absolute path from site root so browser-based renderers (latex.js) can fetch it:
	\includegraphics[width=0.6\textwidth]{/posts/Minimum_Constellation_Size_for_Polar_LEO_Global_Coverage/003.png}
	\caption{Satellite motion and Earth rotation.}
	\label{fig:constellation-example}
\end{figure}

\section{A concrete example}

For: altitude = 500 km; inclination = $90^\circ$; field of view = $30^\circ$, 
We get: 4 orbital planes, 10 satellites per plane, and 40 satellites total.

\begin{figure}[ht]
	\centering
	% Use an absolute path from site root so browser-based renderers (latex.js) can fetch it:
	\includegraphics[width=0.6\textwidth]{/posts/Minimum_Constellation_Size_for_Polar_LEO_Global_Coverage/004.png}
	\caption{Coverage example with 4 orbital planes and 10 satellites per plane.}
	\label{fig:constellation-example}
\end{figure}

\section{Conclusion}

In simple terms, the number of orbital planes directly affects the spatial coverage and regional uniformity of the constellation over the Earth's surface within a given time period, 
while the number of satellites on each orbital plane determines the spatial sampling density of ground observations within the same time period, 
i.e., the granularity of coverage.

\end{document}