\documentclass{article}
\usepackage{graphicx}
% latex.js may not parse \graphicspath reliably; comment it out and reference images by full or relative path
% \graphicspath{{/posts/Minimum_Constellation_Size_for_Polar_LEO_Global_Coverage/}}
\title{Minimum Constellation Size for Polar LEO Global Coverage}
\author{Zi Jyun Maa}
\date{September 2025}

\begin{document}
\maketitle

\section{Introduction}

This document presents an analytical method for determining the minimum size of a LEO polar-orbiting satellite constellation. Closed-form expressions for the required number of orbital planes, satellites per orbit, and total satellite count are derived by modelling the relationships between Earth's rotation, satellite orbital motion, and coverage requirements.

For polar orbits (inclination $90^\circ$), each orbital plane covers all latitudes, but along the equator each plane covers only a strip of width $L_{\text{swath}}$. Within the allowed response time $t_{\max}$, Earth’s rotation causes a maximum equatorial offset of $S^{\text{equ}}_{t_{\max}}$. To ensure seamless global coverage, the spacing between adjacent orbital planes must not exceed $2 \cdot S^{\text{equ}}_{t_{\max}}$.

\section{Definitions}
$R_e$, $T_e$, $t_{\max}$, $L_{\text{swath}}$, $R_o$, $T_o$

\section{Formula Derivation}

Earth’s equatorial linear velocity:
\begin{equation}
v_e = \frac{2\pi R_e}{T_e}
\end{equation}

Equatorial displacement within $t_{\max}$:
\begin{equation}
S^{\text{equ}}_{t_{\max}} = \frac{2\pi R_e}{T_e}\, t_{\max}
\end{equation}

Number of orbital planes:
\begin{equation}
n_{\text{orb}} = \frac{T_e}{2t_{\max}}
\end{equation}

Time for Earth’s rotation to shift by one swath:
\begin{equation}
t_{\text{swath}} = \frac{L_{\text{swath}}T_e}{2\pi R_e}
\end{equation}

Satellite displacement during $t_{\text{swath}}$:
\begin{equation}
S^{\text{sat}}_{t_{\text{swath}}} = \frac{R_o L_{\text{swath}}T_e}{T_o R_e}
\end{equation}

Satellites required per orbit:
\begin{equation}
n_{\text{sat/orb}} = \frac{2\pi R_e T_o}{L_{\text{swath}}T_e}
\end{equation}

Total constellation size:
\begin{equation}
n_{\text{sat}} = \frac{\pi R_e T_o}{L_{\text{swath}} t_{\max}}
\end{equation}

\section{Example}

In case of altitude 500 km, 
inclination 90$^\circ$, 
FOV 30$^\circ$,
Orbital planes: 4
Satellites per plane: 10
Total satellites: 40.

% Example image insertion (use absolute site path or relative file path)
\begin{figure}[ht]
	\centering
	% Use an absolute path from site root so browser-based renderers (latex.js) can fetch it:
	\includegraphics[width=0.8\textwidth]{/posts/Minimum_Constellation_Size_for_Polar_LEO_Global_Coverage/001.png}
	\caption{Constellation schematic (example)}
	\label{fig:constellation-example}
\end{figure}

\section{Analysis}

When the number of orbital planes is fixed, increasing satellites per plane reduces latency until full temporal coverage is achieved. When satellites per plane are fixed, increasing the number of orbital planes significantly improves global coverage. Configurations with four or more planes achieve a valid experiment ratio of 1 when each plane carries 10 satellites. Constellations with fewer than four planes cannot meet global coverage requirements regardless of satellite count.

Scoring metric:
\begin{equation}
\mathrm{score} = \alpha\frac{1}{n} + \beta\frac{1}{t_{2,\mathrm{median}}} + \gamma\ln(r_{\mathrm{sc}})
\end{equation}

\section{Conclusion}

The number of orbital planes determines global coverage and spatial uniformity, while satellites per plane determine ground sampling density. Together, they define the minimum constellation size needed to satisfy the required temporal and spatial coverage constraints.

\end{document}