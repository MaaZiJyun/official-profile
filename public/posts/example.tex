\documentclass{article}
\title{Experiment on 90 Degree Constellation}
\author{Zi Jyun Maa}
\date{September 2025}

\begin{document}
\maketitle

\section{Introduction}
This document reports an analytical method for calculating the minimum size of a constellation of LEO polar-orbiting satellites, 
in which precise expressions for the number of orbital planes, 
the number of satellites per orbit and the total number of satellites are obtained by modelling the mathematical relationships between the Earth's rotation, 
the satellite orbital motions and the coverage requirements.
Since the satellite orbits are polar orbits (with an inclination of 90°), 
each orbital plane covers the entire range of latitudes (from the South Pole to the North Pole). 
However, along the equator, 
each orbital plane can only cover a single strip (with a width of $L_{swath}$). 
The key point here is: within the time $t_{max}$, 
due to Earth's rotation, the maximum offset in the equatorial direction between the coverage areas of adjacent orbital planes is $S^{equ}_{t_{max}}$. 
To ensure seamless global coverage, 
the orbital planes must be evenly distributed along the equator, 
such that the equatorial distance between the coverage regions of adjacent orbital planes does not exceed $2 \cdot S^{equ}_{t_{max}}$.

\section{Background}
LaTeX is a typesetting system commonly used for scientific documents. This sample demonstrates a simple workflow where the raw .tex file is fetched by the browser and converted to HTML using a JS LaTeX renderer.

\section{Conclusion}
This is only a demo. For production use you may prefer server-side conversion (Pandoc, TeX→HTML) or prebuilt PDFs.

\end{document}
