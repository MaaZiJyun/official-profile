\documentclass{article}
\usepackage{graphicx}
\title{Experiment on Single Orbit Tasking Duration Minimisation}
\author{Zi Jyun Maa}
\date{May 2025}

\begin{document}
\maketitle

\section{Introduction}

This study explores whether a \textbf{single-orbit satellite constellation} can complete tasks fast enough for real-time applications. At first glance, a single orbit sounds simple and elegant—so we tested if it could actually work.

Our goal is to minimise the tasking duration:
\[
T_{\text{task}} = t_r - t_0,
\]
the time from task creation to final data downlink.

For simplicity, each task includes only three phases:

(A) \textbf{Tasking}: receiving commands from the ground.  
(B) \textbf{Observing}: capturing images of the target.  
(C) \textbf{Downlink}: sending data back to Earth.

Two basic conditions must also be satisfied:

\textbf{Condition 1 (GS Access)}: the satellite must be visible above a $10^\circ$ elevation angle.

\begin{figure}[ht]
	\centering
	\includegraphics[width=0.3\textwidth]{/posts/Experiment_on_Single_Orbit/001.png}
	\caption{Minimum elevation angle required for visibility.}
\end{figure}

\textbf{Condition 2 (RoI Observation)}: the target must fall inside 90\% of the satellite's field of view.

\section{Experiment Setup}

We conducted a full-factorial experiment using four variables:

\begin{itemize}
    \item Number of satellites: 1, 6, 12, 24  
    \item Number of ground stations: 1, 2, 4, 6  
    \item Ground station latitude band: low / mid / high  
    \item RoI latitude band: low / mid / high  
\end{itemize}

This results in $4 \times 4 \times 3 \times 3 = 144$ scenario combinations.

\begin{table}[h!]
\centering
\begin{tabular}{c|c|p{8cm}}
\hline
\textbf{Factor} & \textbf{Values} & \textbf{Explanation} \\
\hline
$N_{\text{sat}}$ & \{1, 6, 12, 24\} & Satellites in a single polar orbit \\
$N_{\text{gs}}$ & \{1, 2, 4, 6\} & Ground stations evenly spaced in longitude \\
$\phi_{\text{GS}}$ & \([0^\circ, 30^\circ), [30^\circ, 60^\circ), [60^\circ, 90^\circ)\) & GS latitude band \\
$\phi_{\text{RoI}}$ & \([0^\circ, 30^\circ), [30^\circ, 60^\circ), [60^\circ, 90^\circ)\) & RoI latitude band \\
\hline
\end{tabular}
\caption{Experimental factors.}
\end{table}

\section{Experiment Implementation}

To avoid bias, each of the 10 experiment batches randomly generates new GS and RoI coordinates (within the selected latitude band). Each batch then tests all 144 parameter combinations.

Each simulation follows this workflow:

\begin{itemize}
    \item Generate orbits (500 km, 98° inclination) and evenly spaced satellites  
    \item Estimate orbit period $T$, number of shots, and step size $d_t = T/n_{shot}$  
    \item Simulate from first possible observation until successful downlink  
    \item Record the total task duration and parameters  
\end{itemize}

\section{Experiment Results}

A total of 1,440 valid simulation rows were collected.

\begin{figure}[ht]
	\centering
	\includegraphics[width=0.9\textwidth]{/posts/Experiment_on_Single_Orbit/002.png}
	\caption{Randomly generated RoI distribution.}
\end{figure}

The satellites operate at 500 km altitude, 90° inclination, with a 30° FoV.  
The orbit period is 5668 s (1.57 h), and each satellite takes 180 shots per cycle.  
The maximum simulation duration is 3.149 hours with a 31.49 s step size.

\begin{table}[h!]
\centering
\begin{tabular}{l|c|c|c|c|c}
\hline
Stage & Base & Successes & Success Rate & Failures & Failure Rate \\
\hline
Tasking    & 1440 & 1113 & 77.3\% & 327 & 22.7\% \\
Observing  & 1113 & 586  & 52.7\% & 527 & 47.3\% \\
Targeting  & 586  & 562  & 95.9\% & 24  & 4.1\%  \\
\hline
Total      & 1440 & 562  & 39\%   & 878 & 61\%   \\
\hline
\end{tabular}
\caption{Performance summary.}
\end{table}

The main source of failure occurs in the observing stage—satellites often cannot view the RoI in time.  
This is largely due to orbital inclination:

\begin{itemize}
    \item Polar satellites sweep north–south and rely on Earth's rotation to cover longitude.  
    \item The Earth takes 24 hours to rotate once, so full coverage also takes 24 hours.  
    \item Our mission requires a response within only two orbital cycles—far too short for polar coverage.  
\end{itemize}

Thus, many failures are not “errors” but \textbf{physical limitations} of polar orbits.

\begin{figure}[ht]
	\centering
	\includegraphics[width=0.6\textwidth]{/posts/Experiment_on_Single_Orbit/003.png}
	\caption{Task period distribution.}
\end{figure}

The task duration histogram shows two main clusters:  
a short group (0–1000 s), a larger group (6000–10,000 s), and a long tail up to 20,000 s.

\section{Conclusion}

\begin{itemize}
    \item High-latitude GS and RoI placements significantly shorten tasking duration.  
    \item Adding more satellites or GS helps, but latitude effects are even stronger.  
    \item A single polar orbit cannot guarantee real-time response.  
    \item A practical solution is to combine polar and low-inclination orbits.  
\end{itemize}

\end{document}
