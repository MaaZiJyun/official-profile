\documentclass{article}
\usepackage{graphicx}
% latex.js may not parse \graphicspath reliably; comment it out and reference images by full or relative path
% \graphicspath{{/posts/Minimum_Constellation_Size_for_Polar_LEO_Global_Coverage/}}
\title{Experiment on Single Orbit Tasking Duration Minimisation}
\author{Zi Jyun Maa}
\date{May 2025}

\begin{document}
\maketitle

\section{Introduction}
Our goal was to create a real-time satellite system, 
one that could complete missions very quickly. 
Initially, we assumed a single-orbit constellation could meet this requirement and did not consider multi-orbit constellations.
Therefore, we conducted an experiment to determine whether a single orbit could achieve the desired tasking duration.

The objective is to minimise the time interval $\Delta t$ from task initiation $t_0$ to data downlink $t_r$, 
subject to constraints on orbital configuration, satellite resources, 
and ground station distribution.

\begin{equation}
\min \, T_{\text{task}}
\end{equation}
where $T_{\text{task}} = t_r - t_0$ is the tasking duration, and $t_0$ is the task initiation time.

We simplified the problem by assuming a task only includes three phases:

(A) Tasking phase: from $t_0$ to $t_l$, when the satellite receives tasking commands from ground station.
(B) Processing phase: from $t_l$ to $t_o$, when the satellite captures data over the target area.
(C) Downlink phase: from $t_o$ to $t_r$, when the satellite transmits data to the ground station.

Additionally, there are two constraints:

\textbf{Condition 1 (GS Access)}: The satellite passes with an elevation angle > 10°. 
The minimum elevation angle required to establish a connection between the terminal and the satellite is considered the minimum elevation angle for visibility, 
denoted as $\theta_v = 10^\circ$

\begin{figure}[ht]
	\centering
	% Use an absolute path from site root so browser-based renderers (latex.js) can fetch it:
	\includegraphics[width=0.3\textwidth]{/posts/Experiment_on_Single_Orbit/001.png}
	\caption{The minimum elevation angle for visibility}
	\label{fig:the-minimum-elevation-angle}
\end{figure}

\textbf{Condition 2 (RoI Observation)}: The centre of the satellite field of view (or field of view coverage) contains the RoI (coverage > 90\%).

\section{Experiment Setup}

\begin{table}[h!]
\centering
\begin{tabular}{c|c|p{8cm}}
\hline
\textbf{Factor} & \textbf{Experimental Value} & \textbf{Explanation} \\
\hline
$N_{\text{sat}}$ & \{1, 6, 12, 24\} &
Total number of satellites in a single circular polar orbit \\

$N_{\text{gs}}$ & \{1, 2, 4, 6\} &
Number of ground stations evenly distributed in longitude \\

$\phi_{\text{GS}}$ & \([0^\circ, 30^\circ), [30^\circ, 60^\circ), [60^\circ, 90^\circ)\) &
Latitude range where the ground stations are located (low / mid / high latitudes) \\

$\phi_{\text{RoI}}$ & \([0^\circ, 30^\circ), [30^\circ, 60^\circ), [60^\circ, 90^\circ)\) &
Latitude range where the Region of Interest (RoI) is located (low / mid / high latitudes) \\
\hline
\end{tabular}
\caption{Experimental factors and their corresponding values.}
\end{table}


\end{document}